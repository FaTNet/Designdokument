\section{Projektziel}
\label{Projektziel}
Das Projektziel ist es, ein fehlertolerantes System zu entwerfen, zu bauen und dieses zu präsentieren.
Zur Veranschaulichung der Fehlertoleranz wird ein Demonstrationssystem (im folgenden Demonstrator genannt) entworfen, in welches Fehler injiziert und anschließend von diesem behandelt werden.
Dabei sei angemerkt, dass das System nicht gegen jeden Fehler tolerant ist, sondern nur gegen diejenigen, die für die Demonstration eingeplant werden.
Die Gestaltung des Systems wird anhand seines Missionszieles durchgeführt.

Folgende abstrakte Fehler sollen im Rahmen der Fehlertoleranz kompensiert werden können:

\begin{itemize}
	\item Teilausfall der Kommunikation
	\item Ausfall von Motoren
	\item Ausfal/St"orung von optischen Sensoren
	\item Ausfall von anderen Sensoren
	\item Ausfall von Mikrocontrollern mit unterschiedlichen Aufgaben
\end{itemize}

%Diese Fehler wurden nach den folgenden Kriterien ausgesucht:

%\begin{itemize}
%	\item Darstellbarkeit
%	\item Plausibilität
%	\item Injizierbarkeit
%	\item Auswirkung
%	\item Behhebarkeit
%\end{itemize}

Ein Ausfall bedeutet das die betroffene Komponente vollst"andig inaktiv ist, eine St"orung bedeutet das die Komponente falsche Daten liefert.  
Alle Fehler können tatsächlich auftreten, lassen sich leicht und anschaulich injizieren und haben eine angemessene Auswirkung auf das System.