\section{Laufzeit und Organisation}
\label{LauzeitUndOrganisation} 
Das Projekt beginnt im Wintersemester 2012 (15.10.2012) und hat eine Laufzeit von zwei Semestern.

Während der gesamten Laufzeit sind die Projektmitglieder selbst für ihren Fortschritt zuständig. Zwei der Projektmitglieder werden zu Anfang bestimmt, um sich fortan um allgemeine organisatorische Inhalte zu kümmern.
   
Da jedes Mitglied andere Kompetenzen und Interessen vorzuweisen hat, kann in erster Linie jeder selbst bestimmen, an welchen Teilprojekten er/sie mitarbeiten möchte. Dazu werden Arbeitspakete definiert und Personen zugeordnet.
   
Des weiteren gibt es ein wöchentliches Gesamtgruppentreffen, um alle Arbeiten der Woche vorzustellen und das weitere Vorgehen gemeinsam zu planen. Diejenigen Mitglieder, welche zu spät oder gar nicht erscheinen, haben sich vorher abzumelden. Zusätzlich findet regelmäßig ein Treffen mit den Betreuern statt.

Damit alle Projektteilnehmer auf dem Laufenden bleiben, ist ein Projektverwaltungstool (FB3 Redmine) eingerichtet, das ein Wiki, ein Versionsverwaltungssystem sowie eine Ablage für Dokumente und Gantt-Diagramme bereitstellt. Dort findet sich auch das Dokument Arbeitsablauf.vsd in welchem die Arbeitspakete, die Verantwortlichen und die geschätzte Bearbeitungszeit eingetragen sind.
